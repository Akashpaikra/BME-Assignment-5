\documentclass[12pt]{article}
\usepackage[english]{babel}
\usepackage{natbib}
\usepackage{url}
\usepackage[utf8x]{inputenc}
\usepackage{amsmath}
\usepackage{graphicx}
\graphicspath{{NITLOGO/}}
\usepackage{parskip}
\usepackage{fancyhdr}
\usepackage{vmargin}
\setmarginsrb{3 cm}{2.5 cm}{3 cm}{2.5 cm}{1 cm}{1.5 cm}{1 cm}{1.5 cm}

\title{EMERGING TECHNOLOGIES IN HEALTHCARE}\newline \\\\\\				
\author{21111006}								
\date{28 JAN 2022}						
\makeatletter
\let\thetitle\@title
\let\theauthor\@author
\let\thedate\@date
\makeatother

\pagestyle{fancy}
\fancyhf{}
\rhead{\theauthor}
\lhead{\thetitle}
\cfoot{\thepage}

\begin{document}
\begin{titlepage}
	\centering
    \includegraphics[scale = 0.22]{NITLOGO}\\[1.0 cm]	
    \textsc{\LARGE National Institute Of Technology \newline\\\\ RAIPUR}\\[2.0 CM]
    
	\textsc{\Large ASSIGNMENT 05}\\[0.5 cm]				% Course Code
	\rule{\linewidth}{0.4 mm} \\[0.4 cm]
	{ \huge \bfseries \thetitle}\\
	\rule{\linewidth}{0.4 mm} \\[1.5 cm]
	
	\begin{minipage}{0.6\textwidth}
		\begin{flushleft} \large
			\emph{Submitted To:}\\
			Mr. Saurabh Gupta\\
            Department Of Basic Biomedical Engineering\\
			\end{flushleft}
			\end{minipage}~
			\begin{minipage}{0.4\textwidth}
            
			\begin{flushright} \large
			\emph{Submitted By :}\\
			Akash Paikra\\
            21111006\\
		\end{flushright}
        
	\end{minipage}\\[2 cm]
\end{titlepage}

\tableofcontents
\pagebreak







\section{INTRODUCTION:}

Advances in digital healthcare technologies, such as artificial intelligence, virtual reality, 3D printing, robots, and nanotechnology, are altering the future of healthcare right before our eyes. We must keep up with current events in order to be able to influence technology rather than the other way around. Working hand-in-hand with technology is the future of healthcare, and healthcare workers must embrace innovative healthcare technologies to stay relevant in the coming years.

\subsection{VIRTUAL REALITY:}

Virtual reality (VR) is transforming the lives of both patients and doctors. In the future, you might be able to witness procedures as if you were the surgeon, or you might be able to go to Iceland or back home while laying in a hospital bed.




Virtual reality is being utilised to train future surgeons as well as to practise surgeries by current surgeons. Companies like Osso VR and ImmersiveTouch have developed and delivered such software programmes, which are currently in use with promising results. According to a recent Harvard Business Review research, virtual reality-trained surgeons outperformed their traditional-trained colleagues by 230 percent. In addition, the former were speedier and more precise when executing surgical procedures.





Patients benefit from the technology, which has been shown to be effective in pain control. Virtual reality headsets are being given to women to help them picture relaxing surroundings while in labour. When patients with gastrointestinal, cardiac, neurological, and post-surgical pain used virtual reality to distract them from uncomfortable stimuli, their pain levels decreased. Patients undergoing surgery reported reduced discomfort and anxiety, as well as a better overall hospital experience, according to a 2019 pilot research.


\subsection{3D-PRINTING:}



3D printing has the potential to revolutionise healthcare in every way. We can now print biotissues, artificial limbs, medications, blood vessels, and the list goes on, and it is likely that we will continue to do so in the future.

Researchers at the Rensselaer Polytechnic Institute in Troy, New York, devised a method to 3D-print living skin and blood arteries in November of this year. This breakthrough is critical for burn victims who require skin grafts. NGOs such as Refugee Open Ware and Not Impossible, which 3D-print prosthetics for refugees from war-torn places, are also assisting people in need.


This technique also benefits the pharmaceutical business. Since 2015, the FDA has approved 3D-printed medications, and researchers are now working on 3D-printed "polypills." These contain multiple layers of medications to aid patients in sticking to their treatment plans.



\subsection{MEDICAL TRICORDER:}

When it comes to gadgets and rapid answers, every healthcare practitioner has a dream: to have one all-powerful and omnipotent instrument that can diagnose and analyse any condition. It even appeared in Star Trek as the medical tricorder, though only on screen. When Dr. McCoy scanned a patient with his tricorder, the portable, hand–held device displayed vital signs, other parameters, and a diagnosis right away. For doctors, it was the Swiss Army knife.


With the rapid advancement of healthcare technology, we now live in a world where gadgets that were once only a dream of sci-fi aficionados' imaginations are now a reality! One such palm-sized device is the Viatom CheckMe Pro, which can assess ECG, heart rate, oxygen saturation, temperature, blood pressure, and more! Other firms are working on comparable devices, such as the MedWand, which, in addition to tracking multiple vital factors, has a camera for telemedicine. Then there's BioIntelliSense's FDA-approved BioSticker, which, despite its small size and thinness, can monitor a variety of characteristics such as breathing rate, heart rate, skin temperature, body position, activity levels, sleep status, gait, and more.


We'll get there eventually, even though the currently available items are a long way from the tricorder. High–powered microscopes with cellphones will be used to analyse swab samples and images of skin lesions, for example. Sensors might identify DNA anomalies, as well as antibodies and particular proteins. An electronic nose, an ultrasonic probe, or almost anything else currently available might be connected to a smartphone and used to enhance its capabilities. And we need to prepare for it!



\subsection{NANOTECHNOLOGY:}

We are at the beginning of the nanomedicine era. Nanoparticles and nanodevices, I believe, will soon be used as precise medication delivery systems, cancer therapy tools, and miniature surgeons.

Researchers at the Max Planck Institute developed scallop-like microbots that could swim through your physiological fluids as early as 2014. Small, smart tablets, such as the PillCam, are already being used for noninvasive, patient-friendly colon inspections. In late 2018, MIT researchers developed a remotely controlled electronic pill that can communicate diagnostic information or release medications in response to smartphone orders.


Smart patches, which are based on nanotechnology, are also making progress. Grapheal, a firm located in France, exhibited their smart patch at CES 2020, which enables for continuous wound monitoring and even stimulates wound healing thanks to its graphene core.

We will see more practical examples of nanotechnology in medicine as technology advances. Future PillCams might also take biopsy samples for additional study, while nano-surgeons could become a reality thanks to remote-controlled capsules.

\subsection{ROBOTICS:}
 
 
 
 Robotics is one of the most fascinating and rapidly developing disciplines in healthcare, with advancements ranging from robot companions to surgical robots, pharmabotics, disinfection robots, and exoskeletons.

Exoskeletons had a fantastic year in 2019. It saw the first exoskeleton-assisted surgery in Europe, as well as a tetraplegic guy who could control an exoskeleton with his mind! There are numerous other uses for these sci-fi costumes, ranging from assisting nurses to lifting elderly patients to assisting patients with spinal cord injuries.


In healthcare, robot companions can help alleviate loneliness, manage mental health difficulties, and even assist children with chronic illnesses. Jibo, Pepper, Paro, and Buddy are all instances of existing robots. Touch sensors, cameras, and microphones are included in some of them so that their owners can interact with them. For example, an Australian business called ikki is helping children with chronic illnesses keep track of their prescriptions, temperature, and respiration rate while entertaining them with music and stories.

















































\end{document}